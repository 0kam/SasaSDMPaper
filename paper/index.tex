% Options for packages loaded elsewhere
% Options for packages loaded elsewhere
\PassOptionsToPackage{unicode}{hyperref}
\PassOptionsToPackage{hyphens}{url}
\PassOptionsToPackage{dvipsnames,svgnames,x11names}{xcolor}
%
\documentclass[
  authoryear,
  preprint]{elsarticle}
\usepackage{xcolor}
\usepackage{amsmath,amssymb}
\setcounter{secnumdepth}{5}
\usepackage{iftex}
\ifPDFTeX
  \usepackage[T1]{fontenc}
  \usepackage[utf8]{inputenc}
  \usepackage{textcomp} % provide euro and other symbols
\else % if luatex or xetex
  \usepackage{unicode-math} % this also loads fontspec
  \defaultfontfeatures{Scale=MatchLowercase}
  \defaultfontfeatures[\rmfamily]{Ligatures=TeX,Scale=1}
\fi
\usepackage{lmodern}
\ifPDFTeX\else
  % xetex/luatex font selection
\fi
% Use upquote if available, for straight quotes in verbatim environments
\IfFileExists{upquote.sty}{\usepackage{upquote}}{}
\IfFileExists{microtype.sty}{% use microtype if available
  \usepackage[]{microtype}
  \UseMicrotypeSet[protrusion]{basicmath} % disable protrusion for tt fonts
}{}
\makeatletter
\@ifundefined{KOMAClassName}{% if non-KOMA class
  \IfFileExists{parskip.sty}{%
    \usepackage{parskip}
  }{% else
    \setlength{\parindent}{0pt}
    \setlength{\parskip}{6pt plus 2pt minus 1pt}}
}{% if KOMA class
  \KOMAoptions{parskip=half}}
\makeatother
% Make \paragraph and \subparagraph free-standing
\makeatletter
\ifx\paragraph\undefined\else
  \let\oldparagraph\paragraph
  \renewcommand{\paragraph}{
    \@ifstar
      \xxxParagraphStar
      \xxxParagraphNoStar
  }
  \newcommand{\xxxParagraphStar}[1]{\oldparagraph*{#1}\mbox{}}
  \newcommand{\xxxParagraphNoStar}[1]{\oldparagraph{#1}\mbox{}}
\fi
\ifx\subparagraph\undefined\else
  \let\oldsubparagraph\subparagraph
  \renewcommand{\subparagraph}{
    \@ifstar
      \xxxSubParagraphStar
      \xxxSubParagraphNoStar
  }
  \newcommand{\xxxSubParagraphStar}[1]{\oldsubparagraph*{#1}\mbox{}}
  \newcommand{\xxxSubParagraphNoStar}[1]{\oldsubparagraph{#1}\mbox{}}
\fi
\makeatother


\usepackage{longtable,booktabs,array}
\usepackage{calc} % for calculating minipage widths
% Correct order of tables after \paragraph or \subparagraph
\usepackage{etoolbox}
\makeatletter
\patchcmd\longtable{\par}{\if@noskipsec\mbox{}\fi\par}{}{}
\makeatother
% Allow footnotes in longtable head/foot
\IfFileExists{footnotehyper.sty}{\usepackage{footnotehyper}}{\usepackage{footnote}}
\makesavenoteenv{longtable}
\usepackage{graphicx}
\makeatletter
\newsavebox\pandoc@box
\newcommand*\pandocbounded[1]{% scales image to fit in text height/width
  \sbox\pandoc@box{#1}%
  \Gscale@div\@tempa{\textheight}{\dimexpr\ht\pandoc@box+\dp\pandoc@box\relax}%
  \Gscale@div\@tempb{\linewidth}{\wd\pandoc@box}%
  \ifdim\@tempb\p@<\@tempa\p@\let\@tempa\@tempb\fi% select the smaller of both
  \ifdim\@tempa\p@<\p@\scalebox{\@tempa}{\usebox\pandoc@box}%
  \else\usebox{\pandoc@box}%
  \fi%
}
% Set default figure placement to htbp
\def\fps@figure{htbp}
\makeatother





\setlength{\emergencystretch}{3em} % prevent overfull lines

\providecommand{\tightlist}{%
  \setlength{\itemsep}{0pt}\setlength{\parskip}{0pt}}



 
\usepackage[]{natbib}
\bibliographystyle{elsarticle-harv}


\newpageafter{abstract}
\usepackage{lineno}
\linenumbers
\makeatletter
\@ifpackageloaded{caption}{}{\usepackage{caption}}
\AtBeginDocument{%
\ifdefined\contentsname
  \renewcommand*\contentsname{Table of contents}
\else
  \newcommand\contentsname{Table of contents}
\fi
\ifdefined\listfigurename
  \renewcommand*\listfigurename{List of Figures}
\else
  \newcommand\listfigurename{List of Figures}
\fi
\ifdefined\listtablename
  \renewcommand*\listtablename{List of Tables}
\else
  \newcommand\listtablename{List of Tables}
\fi
\ifdefined\figurename
  \renewcommand*\figurename{Figure}
\else
  \newcommand\figurename{Figure}
\fi
\ifdefined\tablename
  \renewcommand*\tablename{Table}
\else
  \newcommand\tablename{Table}
\fi
}
\@ifpackageloaded{float}{}{\usepackage{float}}
\floatstyle{ruled}
\@ifundefined{c@chapter}{\newfloat{codelisting}{h}{lop}}{\newfloat{codelisting}{h}{lop}[chapter]}
\floatname{codelisting}{Listing}
\newcommand*\listoflistings{\listof{codelisting}{List of Listings}}
\makeatother
\makeatletter
\makeatother
\makeatletter
\@ifpackageloaded{caption}{}{\usepackage{caption}}
\@ifpackageloaded{subcaption}{}{\usepackage{subcaption}}
\makeatother
\journal{Remote Sensing in Ecology and Conservation}
\usepackage{bookmark}
\IfFileExists{xurl.sty}{\usepackage{xurl}}{} % add URL line breaks if available
\urlstyle{same}
\hypersetup{
  pdftitle={Time-lapse camera observations enable fine-grained detection and prediction of alpine vegetation change under earlier snowmelt},
  pdfkeywords={alpine vegetation, habitat suitability modeling, climate
change},
  colorlinks=true,
  linkcolor={blue},
  filecolor={Maroon},
  citecolor={Blue},
  urlcolor={Blue},
  pdfcreator={LaTeX via pandoc}}


\setlength{\parindent}{6pt}
\begin{document}

\begin{frontmatter}
\title{Time-lapse camera observations enable fine-grained detection and
prediction of alpine vegetation change under earlier snowmelt}
\author[1]{Ryotaro Okamoto%
%
}
 \ead{okamoto.ryotaro@nies.go.jp} 
\author[1]{Hiroyuki Oguma%
%
}
 \ead{oguma@nies.go.jp} 
\author[2]{Reiko Ide%
%
}
 \ead{ide.reiko@nies.go.jp} 

\affiliation[1]{organization={Biodiversity Division, National Institute
for Environmental Studies},,postcodesep={}}
\affiliation[2]{organization={Earth System Division, National Institute
for Environmental Studies},,postcodesep={}}

\cortext[cor1]{Corresponding author}



        
\begin{abstract}
Alpine ecosystems are highly vulnerable to climate change, yet tools for
detecting and predicting fine‑scale vegetation change remain limited. In
the northern Japanese Alps, dwarf bamboo (\emph{Sasa} spp.) is expanding
rapidly and threatens alpine plant diversity. Here, we integrated
ground‑based time‑lapse cameras with habitat suitability models (HSMs)
to quantify past and future \emph{Sasa} expansion at 1 m spatial
resolution. Using time‑lapse images collected from 2012 to 2021, we
generated automated vegetation classification maps and estimated
pixel‑level snowmelt timing. Comparison between 2012 and 2021 showed a
48\% increase in \emph{Sasa} cover. On average, snowmelt timing advanced
by 8.6 days over the past decade. Variable‑importance analysis of HSMs
identified distance from the prior \emph{Sasa} distribution and the
snowmelt day of year (DOY) as the strongest predictors. Under continued
advancement of snowmelt, projections indicate further \emph{Sasa}
expansion by 2030. This study demonstrates that high‑frequency,
high‑resolution time‑lapse imagery combined with HSMs can predict
vegetation change with fine spatial detail. The framework offers
actionable guidance for conservation, including prioritizing areas for
\emph{Sasa} cutting and selecting monitoring sites. The approach is
generalizable to other invasive species, providing a scalable tool to
assess climate impacts and inform biodiversity conservation in
vulnerable alpine ecosystems.
\end{abstract}





\begin{keyword}
    alpine vegetation \sep habitat suitability modeling \sep 
    climate change
\end{keyword}
\end{frontmatter}
    

\section{Introduction}\label{introduction}

Alpine ecosystems are among the most vulnerable to climate change
(\citet{IPCCSR2019HM}). Numerous studies have documented vegetation
shifts in these environments (\citet{Gottfried2012NatClimChange}),
including changes in timberline positions and alpine vegetation
distributions. These shifts vary across regions, driven by complex
interactions between geological and biological factors
(\citet{Grabherr2010GeoComp}, \citet{Klanderud2005JEcol}).

A major driver of vegetation change is the presence of competitively
dominant species, which can accelerate community shifts by outcompeting
other plants. For instance, grass communities in snowbeds
(\citet{Grabherr2003book}) and shrub communities in alpine grasslands
(\citet{Dullinger2003AAA}) strongly influence surrounding species. In
Japanese alpine ecosystems, dwarf bamboo (\emph{Sasa} spp., hereafter
\emph{Sasa}) exemplifies such dominance, expanding vigorously and often
excluding rare alpine plants. \citet{Kudo2011EcoEvo} demonstrated that
earlier snowmelt associated with climate change promotes \emph{Sasa}
invasion into snowbed communities, reducing plant species richness by
75\%. Conversely, vegetation recovery following \emph{Sasa} removal has
been documented (\citet{Kudo2017AlpBot}), emphasizing the importance of
high-resolution monitoring and predictive modeling of \emph{Sasa}
expansion for conservation management.

Because alpine regions are remote and topographically rugged,
traditional vegetation surveys are often costly and logistically
demanding. Moreover, frequent cloud cover and the complex, mosaic-like
structure of alpine vegetation make satellite-based observations
particularly challenging. To address these challenges, we previously
developed a method for generating high-resolution vegetation
classification maps using ground-based time-lapse cameras
(\citet{Okamoto2024RSEC}). This method utilizes temporal changes in
vegetation color captured by time-lapse imagery for classification and
applies precise georectification to align ground-based photographs
spatially.

In this study, we applied this method to quantify \emph{Sasa} expansion
into alpine meadows on the western slope of Mt. Tateyama in the northern
Japanese Alps. In the nearby Murodo-daira plateau, \emph{Sasa} species
(\emph{Sasa kurilensis}, \emph{Sasa spiculosa}, \emph{Sasa senanensis},
and \emph{Sasa palmata}) expanded by 44--260\% between 1977 and 2015
(\citet{Yoshida2016SasaTateyama}). However, prior studies relied on
field surveys or manual interpretation of aerial photographs, limiting
their spatial coverage. By leveraging automated processing of time-lapse
imagery, we produced a large-scale, high-resolution distribution map of
\emph{Sasa} across the study area.

A key advantage of time-lapse cameras is their capacity to capture
frequent, high-resolution records of snowmelt timing---an essential
driver of alpine vegetation dynamics. Leveraging this capability, we
modeled \emph{Sasa} distributions derived from time-lapse imagery using
environmental variables such as snowmelt timing and topographic factors.
We then predicted expansion patterns under scenarios of earlier
snowmelt. Habitat suitability models (HSMs)---statistical frameworks
that relate species occurrence to environmental variables---are widely
used to elucidate habitat preferences (\citet{Guisan1998JVS}), guide
conservation planning (\citet{Larson2004EcolModel}, \citet{Xue2021GEC}),
forecast the spread of invasive species (\citet{Gallien2010DD},
\citet{Gallien2012GEB}), and assess range shifts or extinction risks
under climate change (\citet{Thomas2004Nature}, \citet{Bakkenes2002GCB},
\citet{Amagai2022AVS}).

We constructed HSMs using observed \emph{Sasa} distribution and its
changes to address two primary questions: (1) whether snowmelt timing
serves as a critical determinant of \emph{Sasa} distribution on Mt.
Tateyama, and (2) how \emph{Sasa} distribution may shift under future
conditions of earlier snowmelt. High-resolution predictions of
\emph{Sasa} expansion can substantially support conservation practices
by identifying priority areas for \emph{Sasa} removal and improving
long-term monitoring strategies.

This study provides the first demonstration of detecting and predicting
vegetation distribution changes using ground-based time-lapse imagery.
Our results demonstrate that time-lapse cameras, with their high
temporal frequency and spatial resolution, offer a powerful means for
understanding and forecasting vegetation dynamics in alpine ecosystems.

\section{Materials and Methods}\label{materials-and-methods}

Figure~\ref{fig-overview} presents an overview of the workflow. Using
time-lapse imagery from the northern Japanese Alps, we quantified
\emph{Sasa} expansion and measured the snowmelt day of year (DOY).
\emph{Sasa} expansion was assessed by comparing vegetation
classification maps derived from imagery captured in 2012 and 2021,
while snowmelt DOY was determined from imagery spanning 2011--2021.
Based on the resulting \emph{Sasa} distribution and its changes,
together with snowmelt DOY and topographic features derived from a
digital elevation model (DEM), we constructed habitat suitability models
(HSMs) to predict \emph{Sasa} distribution and expansion. Finally, we
incorporated future snowmelt DOY projected from our observations into
the HSMs to generate detailed forecasts of future \emph{Sasa} invasion.

\begin{figure}

\centering{

\includegraphics[width=1\linewidth,height=\textheight,keepaspectratio]{index_files/figure-pdf/fig-overview-1.png}

}

\caption{\label{fig-overview}Overview of the proposed method.\\
We used time-lapse imagery and a DEM for constructing HSMs for
predicting the \emph{Sasa} distribution.}

\end{figure}%

\subsection{Image acquisition}\label{sec-image-acquisition}

To monitor the distribution of \emph{Sasa}, we used images captured by a
digital time-lapse camera (EOS 5D MK2, Canon Inc., 21 megapixels)
installed by the National Institute for Environmental Studies on the
Murodo-sanso mountain lodge (approximately 2,450 m a.s.l.) in the
northern Japanese Alps. Since 2010, the camera has documented snowmelt
and vegetation dynamics on the western slope of Mt. Tateyama
(2,350--3,015 m), taking photographs hourly from 6:00 to 19:00 between
April and November. For details of the installation and operation, see
\citet{Okamoto2024RSEC}. In this study, we used images taken in
September--October of 2012 and 2021 for quantifying \emph{Sasa}
expansion.

A major advantage of time-lapse cameras is their ability to capture
fine-scale, daily changes in snow cover. Frequent cloud cover and the
steep topography of alpine regions often hinder continuous satellite
observations, making ground-based time-lapse imagery particularly
valuable. Because earlier snowmelt is considered a primary driver of
\emph{Sasa} expansion (\citet{Kudo2011EcoEvo}), high-frequency and
high-resolution data on snowmelt timing are essential for understanding
and predicting its spread. Therefore, we also used imagery acquired from
April to August between 2011 and 2021 to calculate the snowmelt day of
year (DOY) for each pixel. Snowmelt detection followed a workflow based
on Otsu's binarization method (\citet{otsu_binarization}) as described
in \citet{IdeOguma2013EcolInfom}. The extracted snowmelt DOYs were
subsequently incorporated into the habitat suitability models (HSMs).

\subsection{Automated production of vegetation classification
maps}\label{automated-production-of-vegetation-classification-maps}

Vegetation classification maps were produced using the method developed
by \citet{Okamoto2024RSEC}, which consists of three main steps: image
alignment, vegetation classification, and automated georectification.

\begin{itemize}
\tightlist
\item
  \textbf{Image alignment}: All images from 2012 and 2021 were aligned
  to enable precise temporal comparisons. Following
  \citet{Okamoto2024RSEC}, we applied a local-feature-based automated
  alignment approach.\\
\item
  \textbf{Vegetation classification}: Temporal changes in leaf color
  during autumn were analyzed using a recurrent neural network (RNN),
  which classified each pixel into seven categories: Dwarf Pine
  (\emph{Pinus pumila}), Dwarf Bamboo (\emph{Sasa} spp.), Rowans
  (\emph{Sorbus sambucifolia}, \emph{S. matsumurana}), Maple (\emph{Acer
  tschonoskii}), Montane Alder (\emph{Alnus viridis} subsp.
  \emph{maximowiczii}), Other Vegetation (alpine shrubs and herbaceous
  plants), and No Vegetation.\\
\item
  \textbf{Automated georectification}: We automatically extracted ground
  control points (GCPs) from existing orthorectified aerial photographs
  and a digital elevation model (DEM), estimated camera parameters
  (orientation, focal length, and lens distortion), and transformed the
  images into georeferenced data at 1 m spatial resolution using the
  Python package \texttt{alproj} (https://github.com/0kam/alproj). The
  aerial photographs and DEM used for georectification were identical to
  those in \citet{Okamoto2024RSEC}.
\end{itemize}

\subsection{\texorpdfstring{Quantification of \emph{Sasa}
expansion}{Quantification of Sasa expansion}}\label{quantification-of-sasa-expansion}

To detect nine-year changes in \emph{Sasa} distribution, we overlaid
vegetation classification maps from 2012 and 2021
(Figure~\ref{fig-vege12-21}). We calculated the area occupied by
\emph{Sasa} in each year and quantified its expansion by comparing the
two classification maps. Because \emph{Sasa} can remain beneath shrub
canopies, shrub encroachment and growth may cause \emph{Sasa} to become
undetectable in time-lapse imagery. Consequently, areas that appear to
have decreased in \emph{Sasa} cover may not represent true declines. To
avoid underestimating expansion due to this limitation, we defined the
primary expansion metric as the area of transitions from non-\emph{Sasa}
to \emph{Sasa} (i.e., expansion area).

\begin{figure}

\centering{

\includegraphics[width=1\linewidth,height=\textheight,keepaspectratio]{index_files/figure-pdf/fig-vege12-21-1.png}

}

\caption{\label{fig-vege12-21}Vegetation classification results for 2012
and 2021.\\
Vegetation classification maps for 2012 (left) and 2021 (right).
\emph{Sasa} expansion was quantified by comparing the spatial extent of
its distribution between the two maps.}

\end{figure}%

\subsection{Habitat suitability
modeling}\label{habitat-suitability-modeling}

To predict future \emph{Sasa} expansion and understand its driver, we
constructed habitat suitability models (HSMs). \emph{Sasa} forms complex
rhizome systems and extensive clonal networks
(\citet{Suyama2000MolEcol}). While its expansion primarily occurs
through rhizome elongation, seed dispersal can also take place after
mass or partial flowering events (\citet{Miyazaki2009JPlantRes};
\citet{Mizuki2014PLOS}; \citet{Kudo2011EcoEvo}). As a result,
\emph{Sasa} can potentially expand both gradually through rhizome growth
and over longer distances through seed dispersal.

To capture these contrasting dispersal processes, we developed two
complementary models that predict the \emph{Sasa} distribution observed
in 2021 from environmental conditions and the baseline distribution in
2012. The \textbf{Topography-Based Model (TBM)} incorporates only
topographic variables such as elevation, slope, and snowmelt DOY,
representing potential expansion areas assuming that \emph{Sasa} can
disperse freely via seeds. In contrast, the \textbf{Topography--Distance
Model (TDM)} additionally includes the distance from the 2012
\emph{Sasa} distribution as a predictor, constraining the predicted 2021
distribution to areas near existing populations and thus reflecting the
limited rhizome-based spread typical of \emph{Sasa}. Comparing the two
models allows us to distinguish between the broader potential niche
under seed dispersal and the more localized realized expansion expected
from clonal growth, providing a realistic range of possible future
scenarios.

\subsubsection{Explanatory variables}\label{sec-explanatory}

Both models incorporated elevation, slope, aspect, roughness, the
Terrain Ruggedness Index (TRI), the Topographic Position Index (TPI),
and snowmelt day of year (DOY) as explanatory variables. These
topographic features were derived from a 5 m digital elevation model
(DEM) provided by the Geospatial Information Authority of Japan and
resampled to 1 m resolution to match the vegetation maps.

Snowmelt DOYs were obtained from time-lapse imagery. Although
interannual fluctuations were observed, snowmelt timing showed an
advancing trend over the past decade. To represent this temporal trend,
we fitted a linear regression to per-pixel snowmelt DOYs for each year
from 2011 to 2021 and used the predicted 2021 values as explanatory
variables in both models. The mean regression coefficient was −0.86,
corresponding to an advance of 8.6 days per decade. For future
projections, we extended the same regressions to estimate snowmelt DOYs
for 2030 at each pixel. Although some areas obscured by foreground snow
cover had uncertain snowmelt dates, these areas showed minimal overlap
with the Sasa distribution and thus had little impact on our analysis.
Additional details on snowmelt timing and its prediction are provided in
Section~\ref{sec-supp}. In the TDM, distance from the 2012 \emph{Sasa}
distribution was included as an additional explanatory variable to
represent dispersal limitation.

\subsubsection{Target variable}\label{target-variable}

The TBM was trained to predict the \emph{Sasa} distribution in 2021.
Because many \emph{Sasa} patches present in 2012 persisted through 2021,
pixels located directly within the 2012 distribution (i.e., at 0 m
distance) were excluded from the TDM. Accordingly, the TDM was designed
to predict newly colonized locations that appeared between 2012 and
2021.

\subsubsection{Data sampling and
splitting}\label{data-sampling-and-splitting}

\paragraph{Data sampling}\label{data-sampling}

Because the \emph{Sasa} distribution is relatively limited compared with
other vegetation types (Figure~\ref{fig-vege12-21}), the dataset was
imbalanced. To address this, presence data were retained at 1 m
resolution, whereas absence data were downsampled to 5 m resolution to
reduce dominance by non-\emph{Sasa} pixels. Additionally, areas above
2,560 m---where \emph{Sasa} does not occur---were excluded from the
analysis.

\paragraph{Initial splitting}\label{initial-splitting}

To evaluate model performance, we divided the dataset into 80\% for
training and 20\% for testing. Given the strong spatial autocorrelation
among explanatory variables, we applied spatial block splitting using
the R packages \texttt{spatialsample} \citep{spatialsample} and
\texttt{tidysdm} \citep{tidysdm} to mitigate overfitting effects
(Figure~\ref{fig-spatial-split}, left).

\begin{figure}

\centering{

\includegraphics[width=1\linewidth,height=\textheight,keepaspectratio]{index_files/figure-pdf/fig-spatial-split-1.png}

}

\caption{\label{fig-spatial-split}Visualizations of spatial sampling.\\
The initial split is shown on the left, and the fourfold split used for
cross-validation is shown on the right. Spatial sampling was implemented
during data splitting to reduce the risk of model overfitting caused by
spatial autocorrelation.}

\end{figure}%

\paragraph{Cross-validation}\label{cross-validation}

For hyperparameter tuning and ensemble construction, we performed
fourfold cross-validation using the same spatial block splitting
approach (Figure~\ref{fig-spatial-split}, right) on the training
dataset.

\subsubsection{Model preparation}\label{model-preparation}

\paragraph{Model training and
evaluation}\label{model-training-and-evaluation}

We applied four classification algorithms: gradient boosted trees (GBT;
\citet{Friedman2001GBM}), maximum entropy (MaxEnt;
\citet{Philips2006MaxEnt}), random forest (RF;
\citet{Breiman2001RandForest}), and generalized additive models (GAM).
For all models except GAM, hyperparameters were tuned via grid search
with up to 18 trials. Model performance was evaluated using the True
Skill Statistic (TSS; \citet{Allouche2006TSS}; see
Equation~\ref{eq-tss-formula}), which ranges from 0 to 1, with higher
values indicating greater predictive accuracy. (TP), (FP), and (FN)
denote the numbers of true positives, false positives, and false
negatives, respectively.

\begin{equation}\phantomsection\label{eq-tss-formula}{
\begin{gathered}
recall = \frac{TP}{TP + FN} \\
specificity = \frac{TN}{TN + FP} \\
TSS = recall + specificity - 1 \\
\end{gathered}
}\end{equation}

\paragraph{Model ensembling}\label{model-ensembling}

We constructed ensemble models using the \texttt{blend\_predictions}
function in the R package \texttt{stacks} (\citet{RCranStack}), which
estimates blending coefficients via LASSO regularization to maximize
cross-validated performance. The ensemble TSS was subsequently evaluated
on the independent test dataset.

\paragraph{Variable importance
computation}\label{variable-importance-computation}

For both the TBM and TDM, variable importance was assessed using
permutation loss, defined as the reduction in TSS observed when each
explanatory variable was randomly permuted. This measure quantifies the
relative contribution of each variable to the overall predictive
performance of the model.

\subsubsection{Future prediction and definition of risky
areas}\label{future-prediction-and-definition-of-risky-areas}

Using ensemble models built for each of the TBM and TDM, we predicted
\emph{Sasa} distribution for 2030 based on the estimated snowmelt DOYs
(see Section~\ref{sec-explanatory}). In this framework, predictions from
the TBM represent areas that could become suitable if \emph{Sasa}
disperses via seeds. Under such conditions, \emph{Sasa} is particularly
likely to invade low-height alpine grasslands, which correspond to the
``Other Vegetation'' category in our classification. For conservation
purposes, we therefore defined ``risky areas'' as cells classified as
``Other Vegetation'' in 2021 whose TBM-predicted habitat suitability
(HS) for 2030 exceeded 0.5, and we extracted their spatial distribution
accordingly.

\section{Results}\label{results}

\subsection{Changes in Sasa distribution between 2012 and
2021}\label{changes-in-sasa-distribution-between-2012-and-2021}

The area occupied by \emph{Sasa} increased from \(8,542~m^2\) in 2012 to
\(10,170~m^2\) in 2021, a net gain of \(1,628~m^2\) over nine years
(+19\% relative to 2012). The total expansion area---pixels that
transitioned from non-\emph{Sasa} to \emph{Sasa}---amounted to
\(4,095~m^2\) (+48\% relative to 2012), while the decrease area (from
\emph{Sasa} to non-\emph{Sasa}) was \(2,467~m^2\). Many of these
apparent decreases likely resulted from shrub encroachment and growth.
Under such conditions, \emph{Sasa} can persist beneath shrub canopies
but becomes undetectable in time-lapse imagery. Therefore, we adopted
the expansion area (\(4,095~m^2\)) as the primary metric of change. The
spatial distribution of expansion is shown in
Figure~\ref{fig-sasa-expansion}. The observed +48\% increase over nine
years is consistent with previous findings from nearby regions reporting
up to +260\% expansion over 38 years (\citet{Yoshida2016SasaTateyama}).

\begin{figure}

\centering{

\pandocbounded{\includegraphics[keepaspectratio]{files/expanded_area.pdf}}

}

\caption{\label{fig-sasa-expansion}Distribution of \emph{Sasa} expansion
between 2012 and 2021.\\
The area occupied by \emph{Sasa} was \(8,542~m^2\) in 2012 (shown in
green) and expanded by \(4,095~m^2\) (shown in yellow) by 2021,
corresponding to a \(48\%\) increase relative to 2012.}

\end{figure}%

\subsection{HSM performance and variable
importance}\label{hsm-performance-and-variable-importance}

The TDM consistently achieved higher TSS values than the TBM. Within the
TBM, GBT and MaxEnt performed best, while within the TDM, GBT, MaxEnt,
and RF outperformed GAM. Ensemble TSS on the test dataset was 0.55 for
the TBM and 0.70 for the TDM, indicating clear performance gains from
incorporating distance in addition to topographic features.

\begin{figure}

\centering{

\includegraphics[width=1\linewidth,height=\textheight,keepaspectratio]{index_files/figure-pdf/fig-hsm-performance-1.png}

}

\caption{\label{fig-hsm-performance}Performance metrics of HSMs obtained
through hyperparameter tuning during cross-validation.\\
TSS scores of HSMs obtained through hyperparameter tuning during
cross-validation. The TDM consistently achieved higher TSS values than
the TBM, indicating superior predictive performance. Within the TBM
framework, GBT and MaxEnt models performed best, whereas within the TDM
framework, GBT, MaxEnt, and RF models showed comparably high accuracy.}

\end{figure}%

Permutation-based variable importance
(Figure~\ref{fig-variable-importance}) identified snowmelt DOY as the
most influential predictor in the TBM, whereas distance from the 2012
\emph{Sasa} distribution was the strongest predictor in the TDM,
followed by snowmelt DOY.

\begin{figure}

\centering{

\includegraphics[width=1\linewidth,height=\textheight,keepaspectratio]{index_files/figure-pdf/fig-variable-importance-1.png}

}

\caption{\label{fig-variable-importance}Variable importance for TBM and
TDM.\\
Variable importance based on permutation loss for the TBM and TDM. In
the TBM, snowmelt DOY was the most influential variable, whereas in the
TDM, distance from the 2012 \emph{Sasa} distribution was the most
important, followed by snowmelt DOY.}

\end{figure}%

For the 2021 habitat suitability (HS) maps (Figure~\ref{fig-hs-maps}),
the TBM predicted \(47,253~m^2\) of suitable habitat (HS \textgreater{}
0.5), substantially overestimating the observed distribution
(\(10,170~m^2\)). In contrast, the TDM predicted \(12,766~m^2\), much
closer to observations.

\begin{figure}

\centering{

\includegraphics[width=1\linewidth,height=\textheight,keepaspectratio]{index_files/figure-pdf/fig-hs-maps-1.png}

}

\caption{\label{fig-hs-maps}Predicted habitat suitability of \emph{Sasa}
in 2021.\\
Habitat suitability (HS) maps for 2021 generated by the TBM and TDM. The
maps show areas predicted as suitable for \emph{Sasa} habitation (HS
\textgreater{} 0.5). The TBM predicted \(47,253~m^2\) as suitable,
substantially overestimating the observed distribution of
\(10,170~m^2\). In contrast, the TDM predicted \(12,766~m^2\), closely
matching the observed distribution and demonstrating higher accuracy in
reproducing the actual \emph{Sasa} distribution.}

\end{figure}%

\subsection{Future prediction and risky
areas}\label{future-prediction-and-risky-areas}

From the 2030 HS maps and their differences from 2021
(Figure~\ref{fig-future-prediction}), TBM predicted \(27,049~m^2\) (57\%
of the 2021 TBM suitable area) of newly suitable habitat by 2030, while
TDM predicted \(4,387~m^2\) (34\%). Conversely, of the areas occupied by
\emph{Sasa} in 2021, TBM predicted \(2,257~m^2\) (21\%) to become
unsuitable by 2030, and TDM predicted \(717~m^2\) (7\%) to become
unsuitable.

\begin{figure}

\centering{

\includegraphics[width=1\linewidth,height=\textheight,keepaspectratio]{index_files/figure-pdf/fig-future-prediction-1.png}

}

\caption{\label{fig-future-prediction}Predicted habitat suitability of
Sasa in 2030 and differences from 2021.\\
Habitat suitability (HS) maps for 2030 and differences from the 2021 HS
maps. The TBM predicted \(27,049~m^2\) (57\% of the 2021 suitable area)
as newly suitable for Sasa in 2030, while the TDM predicted
\(4,387~m^2\) (34\%) as newly suitable. These projections indicate a
potential Sasa distribution expansion similar to the past decade.
Additionally, even in areas where Sasa was present in 2021, some regions
are expected to become unsuitable by 2030. The TBM identifies
\(2,257~m^2\) (21\%) as unsuitable, while the TDM identifies \(717~m^2\)
(7\%) as unsuitable. These reductions suggest shifts in habitat
suitability and Sasa distribution due to earlier snowmelt timing.}

\end{figure}%

The spatial distribution of risky areas---cells classified as ``Other
Vegetation'' in 2021 with TBM HS \textgreater{} 0.5 in 2030---is shown
in Figure~\ref{fig-risky-map}.

\begin{figure}

\centering{

\includegraphics[width=1\linewidth,height=\textheight,keepaspectratio]{files/risky_tbm.png}

}

\caption{\label{fig-risky-map}Predicted potential habitats for
\emph{Sasa} (risky areas).\\
Potential habitats were extracted from the TBM predictions to support
future monitoring and conservation planning. ``Risky areas'' were
defined as cells classified as ``Other Vegetation'' in 2021 with a
predicted 2030 habitat suitability (HS) greater than 0.5.}

\end{figure}%

\clearpage

\section{Discussion}\label{discussion}

In this study, we applied the method proposed in our previous work to
quantify the expansion of \emph{Sasa} distribution in the northern
Japanese Alps using time-lapse camera imagery. By integrating these data
into habitat suitability models (HSMs), we successfully assessed the
future risk of \emph{Sasa} expansion at an exceptionally high spatial
resolution of 1 m.

The upward expansion of competitively dominant species such as
\emph{Sasa} poses a serious threat to the conservation of alpine
vegetation. The findings of this study provide a critical foundation for
prioritizing conservation actions and designing long-term monitoring
strategies. In the following section, we discuss the advantages of the
proposed workflow for understanding and conserving alpine vegetation, as
well as its methodological limitations and future applications.

\subsection{\texorpdfstring{Expansion of \emph{Sasa} and its
drivers}{Expansion of Sasa and its drivers}}\label{expansion-of-sasa-and-its-drivers}

Comparison of vegetation classification maps from 2012 and 2021 enabled
us to spatially identify areas of \emph{Sasa} expansion. Incorporating
HSMs further allowed us to investigate the environmental factors driving
this expansion. Variable importance analyses for both the TBM and TDM
(Figure~\ref{fig-variable-importance}) indicated that snowmelt DOY was a
major determinant. Moreover, predictions for 2030, assuming earlier
snowmelt than in 2021, suggested continued \emph{Sasa} expansion. These
results indicate that earlier snowmelt has facilitated \emph{Sasa}
expansion, consistent with previous findings (\citet{Kudo2011EcoEvo})
that link earlier snowmelt to increased soil dryness and extended
growing periods that favor \emph{Sasa} growth.

At the same time, the TBM substantially overestimated \emph{Sasa}
distribution in 2021. Although adding distance as an explanatory
variable improved model accuracy, at least two interpretations of the
overestimation are possible. First, areas predicted as suitable by the
TBM may indeed be habitable, but \emph{Sasa} has not yet colonized them
due to its limited dispersal capacity. In fact, no isolated \emph{Sasa}
patches formed by seed dispersal were observed in the expansion areas
between 2012 and 2021, supporting the idea that low dispersal ability
creates a large gap between potential and realized niches. Second, the
overestimation may reflect missing explanatory variables. In this case,
\emph{Sasa} generally occupies its potential niche, but the TBM
overestimated distribution because it failed to incorporate important
environmental factors. The improved accuracy of the TDM could then be
explained by a strong spatial correlation between the 2012 distribution
and such unmeasured variables. Soil structure or soil moisture, for
instance, could play this role. To distinguish between these
interpretations, the TBM should ideally be extended to include
additional environmental predictors, although obtaining such
comprehensive datasets remains challenging. Alternatively, process-based
models that simulate dispersal dynamics could help clarify whether the
overestimation results from dispersal limitations or omitted
environmental factors.

\subsection{Future prediction and potential
applications}\label{future-prediction-and-potential-applications}

This study demonstrated that combining time-lapse cameras with HSMs
enables exceptionally fine-scale (1 m) prediction of future vegetation
distributions. The integration of time-lapse imagery and HSMs is
unprecedented. In contrast, conventional species distribution models
typically operate at much coarser resolutions (e.g., 1 km), as future
climate projections are generally available only at coarse spatial
scales (e.g., alpine vegetation studies: \citet{Amagai2022AVS}). Here,
by directly observing snowmelt timing---a key determinant of \emph{Sasa}
distribution---using time-lapse cameras, we achieved high-resolution
predictions. Because snowmelt timing requires both frequent and
spatially extensive monitoring, time-lapse cameras are particularly
suited for this purpose.

High-resolution predictions are especially valuable for designing
practical conservation strategies, such as identifying priority areas
for \emph{Sasa} removal or selecting effective monitoring sites. For
example, a prediction at 1 km resolution might suggest that an entire
square kilometer is at risk, which would be unrealistic for management.
In contrast, predictions at 1 m resolution enable more targeted and
efficient conservation planning.

In this study, we predicted future snowmelt timing using a simple linear
regression based on a decade of past data. While this approach provided
a useful proof of concept, it does not represent a realistic long-term
forecast. In the alpine regions, strong seasonal winds and complex
topography cause substantial snow redistribution, making snowmelt
prediction a continuing challenge. Therefore, we did not attempt to
refine future snowmelt predictions here. However, as high-resolution
climate projections become available, the accuracy of \emph{Sasa}
distribution forecasts will also improve. The future expansion presented
here should thus be interpreted not as a literal forecast but as a
scenario based on the assumption of continued earlier snowmelt---an
assumption that should be re-evaluated as understanding of alpine snow
processes advances.

\section{Conclusion}\label{conclusion}

This study demonstrated the effectiveness of integrating time-lapse
cameras with habitat suitability models (HSMs) to detect and predict
changes in vegetation distribution at 1 m resolution. Focusing on the
expansion of \emph{Sasa} in the northern Japanese Alps, we quantified
its distributional changes over the past nine years and identified
earlier snowmelt as a major driver of its spread. Furthermore, our
projections indicate that \emph{Sasa} expansion is likely to continue in
the coming decades.

The expansion of \emph{Sasa} poses a serious threat to alpine plant
diversity, and the framework developed here provides direct applications
for conservation planning. Predictions at 1 m resolution offer practical
guidance for selecting monitoring sites and prioritizing areas for
\emph{Sasa} removal, enabling more efficient allocation of conservation
resources. Beyond \emph{Sasa}, this approach can be extended to other
invasive species, as well as to plant communities highly sensitive to
climate change, demonstrating broad potential for use in alpine regions
and other vulnerable ecosystems worldwide.

In summary, this study highlights the power of combining high-frequency
time-lapse observations with HSM-based modeling to advance the
understanding and prediction of vegetation dynamics. This integrated
framework refines climate impact assessment in alpine ecosystems and
supports the development of practical, conservation-oriented ecology.

\clearpage

\section{Supplementary Information}\label{sec-supp}

\setcounter{figure}{0}
\renewcommand{\figurename}{Fig.}
\renewcommand{\thefigure}{S\arabic{figure}}

\begin{figure}

\centering{

\includegraphics[width=1\linewidth,height=\textheight,keepaspectratio]{index_files/figure-pdf/fig-snowmelt-shift-1.png}

}

\caption{\label{fig-snowmelt-shift}Decadal trend in snowmelt timing
(2011--2021).\\
Changes in snowmelt Day of Year (DOY) relative to 2011 (set as 0). A
total of 1,000 pixels were randomly sampled and plotted. Despite large
interannual fluctuations, an overall trend toward earlier snowmelt is
suggested. Data for 2019 are missing due to construction work at the
mountain lodge where the camera was installed.}

\end{figure}%

\begin{figure}

\centering{

\includegraphics[width=1\linewidth,height=\textheight,keepaspectratio]{index_files/figure-pdf/fig-snowmelt-map-1.png}

}

\caption{\label{fig-snowmelt-map}Predicted snowmelt DOY (linear
regression for each cell).\\
Predicted snowmelt DOY for 2012, 2021, and 2030, derived from cell-wise
linear regression models fitted to annual snowmelt DOY between 2011 and
2021.\\
}

\end{figure}%

\begin{figure}

\centering{

\includegraphics[width=1\linewidth,height=\textheight,keepaspectratio]{index_files/figure-pdf/fig-snowmelt-shift-map-1.png}

}

\caption{\label{fig-snowmelt-shift-map}Changes in predicted snowmelt DOY
(2021 -- 2012; negative values indicate earlier snowmelt).\\
Difference in predicted snowmelt DOY between 2012 and 2021. Most areas
show earlier snowmelt, with clear spatial heterogeneity in the degree of
change.\\
}

\end{figure}%

\clearpage


\renewcommand\refname{References}
\bibliography{export.bib}



\end{document}
